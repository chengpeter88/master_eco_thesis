\documentclass[12pt, a4paper]{article} 
\usepackage{tcolorbox}
\tcbuselibrary{skins, breakable, theorems}
\usepackage{subcaption}
\usepackage{pdflscape} 

\newtcbtheorem{question}{题~(理}%
  {enhanced, breakable,
    colback = white, colframe = cyan, colbacktitle = cyan,
    attach boxed title to top left = {yshift = -2mm, xshift = 5mm},
    boxed title style = {sharp corners},
    fonttitle = \sffamily\bfseries, separator sign = {).~}}{qst}
\input{pre._CJK_thesis}   % 使用自己維護的定義檔
%-----------------------------------------------------------------------------------------------------------------------
% 文章開始
\title{ Teps-b資料統計}
\author{{\SM 鄭仲恒}}
\date{{\TT \today}} 	 
\begin{document}
\maketitle
\fontsize{12}{22 pt}\selectfont
Teps-b統計資料與圖形分析,CP代表原先調查為國中生族群,SH為原先調查高中生族群。文章先由CP核心資料先介紹,再後續介紹SH。

%%%%%%%%%%%%%%%%%%%%%%%%%%%%


\section{CP-核心資料國中生}
調查的資料分佈年份為2009、2013、2014、2019年,其中2014年度為專員實際進行面訪受訪者,其餘3次調查譏為電話訪談資料。
\subsection{2009年度調查-電訪}
\begin{figure}[H]
    \centering
    \includegraphics[width=0.9\textwidth]{\imgdir contact.png}
    \caption{受訪者接觸狀況}\label{pic:contact}

\end{figure}





\end{document}
